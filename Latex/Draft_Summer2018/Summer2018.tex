\documentclass[12pt]{article}

% Packages
\usepackage[utf8]{inputenc}
\usepackage[english]{babel}
\usepackage[top=3cm,bottom=3cm,left=3cm,right=3cm,bindingoffset=0mm]{geometry}
\usepackage{amssymb}
\usepackage{amsmath}
\usepackage{tikz}
\usepackage{graphicx}
\usepackage{comment}
\usepackage{rotating}
\usepackage{float}
\usepackage{natbib}
\usepackage{amsthm}
\usepackage{bbm}
\usepackage{thmtools,thm-restate}
\usepackage{hyperref}
%\usepackage{extsizes}
\usepackage[font=footnotesize,labelfont=bf]{caption}

% New Options
\newtheorem{prop}{Proposition}
\newtheorem{definition}{Definition}[section]
\newtheorem*{remark}{Remark}
\newtheorem{lemma}{Lemma}
\declaretheorem{proposition}
\linespread{1.3}
\raggedbottom
\font\reali=msbm10 at 12pt

% New Commands
\newcommand{\real}{\hbox{\reali R}}
\newcommand{\realp}{\hbox{\reali R}_{\scriptscriptstyle +}}
\newcommand{\realpp}{\hbox{\reali R}_{\scriptscriptstyle ++}}
\newcommand{\R}{\mathbb{R}}
\DeclareMathOperator{\E}{\mathbb{E}}

\title{ICT and Future Productivity:\\Evidence and Theory of a GPT\thanks{Correspondence: Department of Economics, Boston College, 140 Commonwealth Avenue, Chestnut Hill, MA 02467. Email: brianti@bc.edu (Marco Brianti) and gati@bc.edu (Laura Gati).}}
\author{Marco Brianti \\ {\small Boston College} \and Laura Gati \\ {\small Boston College}}
\date{\today}


\begin{document}



\maketitle

\

\

\
%%%%%%%%%%%%%%%%%%%%             ABSTRACT           %%%%%%%%%%%%%%%%%% 
\small{
\abstract{Information and Communication technology (ICT) is able to explain accelerations in productivity in sectors that are ICT users. We employ Structural VARs to investigate the effects of ICT supply shocks on Total Factor Productivity (TFP) and other macroeconomic variables. In response to this sector-specific supply shock relative prices of ICT goods and services immediately fall, ICT investment rises on impact, and TFP displays a significant delayed and persistent increase. In line with theories of ICT as a general-purpose technology, we analyze a two-sector general equilibrium model in order to rationalize previous results and estimate key parameters via impulse-response matching. We conclude that ICT accumulation is able to enhance productivity through a positive spillover effect which takes into account the overall level of diffusion of ICT capital in the economy.}
}
\newpage



\newpage
%%%%%%%%%%%%%%%%%%%%               INTRODUCTION                 %%%%%%%%%%%%%%%%%% 
\section{Introduction}

Although there is large consensus on the importance of productivity as a driver of economic performances, less agreement is on the underlying sources that enhance its growth. For several years most of the business-cycle literature purposely decided to avoid such a question by proxying movements in productivity as random shocks.\footnote{\cite{kydland1982time} and \cite{long1983real} are among the first papers which consider productivity shocks on general equilibrium models.} However, the robust empirical evidence of the slowdown in productivity right before the great recession is summoning the literature to take a step back and devote more attention on the drivers of medium-term productivity growth.\footnote{See \cite{cette2016pre} and \cite{byrne2016does} among others.}

Along with \cite{comin2006medium}, some theoretical contributions rationalize endogenous productivity dynamics by adapting features of endogenous growth models into DSGE models. Following \cite{romer1990endogenous}, most of those papers augment final-good production functions with an expanding composite of intermediate goods produced by the R\&D sector in order to allow for an endogenous rate of adoption of new technologies.\footnote{\cite{bianchi2014growth}, \cite{anzoategui2016endogenous}, and \cite{moran2017innovation} use similar techniques to endogenize growth. In particular, \cite{bianchi2014growth} augment a DSGE  model using a quality ladders model in the vein of \cite{grossman1991quality}. Moreover, \cite{anzoategui2016endogenous} and \cite{moran2017innovation}, similarly to \cite{comin2006medium}, use a model of expanding variety in the vein of \cite{romer1990endogenous}.} Consistent with those previous models, other papers attempt to provide empirical evidence of a slowdown in the productivity of the R\&D sector. Specifically, they show that although research effort is keeping rising, the rate of new ideas and discoveries is slowing down.\footnote{\cite{jones2009burden} and \cite{bloom2017ideas} are two important contributions that highlight those facts.}

Motivated by this wave of research, this paper follows a different path and argue that Information and Communication Technology (hereafter ICT) plays an important role in driving medium-term productivity in sectors that are ICT users. Our contribution is twofold. First, we provide a robust empirical evidence to show that current rises in ICT investment explains significant and persistent increases in future Total Factor Productivity (hereafter TFP). Second, we analyze a standard theoretical framework in order to both motivate and rationalize our empirical results. 

Regarding the empirical section, the idea is to identify technological shocks which are only specific to the ICT sector in a Structural VAR context.\footnote{An interesting paper which is somehow related to our empirical part is \cite{jafari2012impact}. The authors identify ICT shocks as a potential driver of the Iranian business cycle using a completely different identification strategy and obtaining qualitatively different results.} In order to have a reliable identification procedure our multivariate system needs to embody three key variables: TFP, ICT investment (hereafter ICTI), and relative prices (hereafter RP). Importantly, ICTI is defined as the total expenditure in equipment and computer software meant to be used in production for more than an year. Thus, an increase in ICTI has to be though as an ICT capital deepening. Moreover RP is the ratio between prices in the ICT sector over prices in the overall economy. To verify that we are correctly identifying an ICT technology shock, we firstly expect it to be orthogonal to the current productivity of all the other sectors. Since the share of the ICT sector accounts for a negligible part in the whole economy we expect it should have an approximately zero effect on TFP on impact. Moreover, as pointed out by \cite{greenwood1997long} and \cite{fisher2006dynamic}, embodying RP and ICTI is important because we expect that a sectoral technology shock should decrease its relative prices and enhance expenditure in the underlying sector.\footnote{However, as suggested by both \cite{greenwood2000role} and \cite{basu2010sector}, we are aware that conditioning our identifying restrictions only on the direction of RP does not properly measure for technological changes between sectors. This is the main reason why we never impose the direction of RP as an direct identifying condition.} In response to this shock, ICTI rises on impact and remains significant for several quarters. RP persistently and significantly declines for more than two years and TFP, which does not react on impact, rises after few quarters and remains significant and stable for at least 5 years. 

Although our results are robust over different specifications, there is an important critique that our empirical strategy had to carefully take into account, which is the reverse causality due to news on future TFP. A well-taken concern motivated my the news-shock literature is that the positive reaction of ICTI on impact may be triggered to signals related future increases in TFP and not to contemporaneous ICT technological improvements. In other words, our identification strategy may confound a news shock which contemporaneously enhances investment in ICT capital goods. In order to take into account this potential issue, we provide a series of robustness checks where sequentially we firstly identify a news shock in the spirit of \cite{barsky2011news} and subsequently we identify our sectoral ICT shock using the previous identification strategy. Encouragingly, controlling for signals regarding future movements in TFP does not affect any conclusion drown so far. In particular, we can now state even more strongly the causality relation from movements in current ICT technological changes to future TFP. 


Motivated by papers in line with \cite{oliner2000resurgence} and \cite{stiroh2002information}, in order to formally explain which mechanism links current ICT to future TFP, we analyze a 2-sector DSGE model which allows ICT to be the general purpose technology (hereafter GPT) of the whole economy. Both sectoral production functions are fed with three inputs: (i) labor, supplied by households, (ii) hard capital, produced by the final sector, and (iii) ICT capital, produced by the ICT sector. As well-explained by both \cite{basu2003case} and \cite{basu2007information}, a GPT should be able to enhance accelerations in productivity in sectors that are users of the underlying technology. We then interpret ICT as the GPT of the last 30 years of the U.S. economy assuming that exogenous technological changes in the ICT sector are able to affect economy-wide productivity. In particular, when an ICT technology shock arrives, both sectors accumulate ICT capital since it is easier to produce and cheaper to buy. This ICT capital deepening consequently enhances the productivity of both sectors by means of a spillover. Since the purpose of ICT capital is to improve information sharing, the quality and speed of communication must mainly depend on the diffusion of these technologies among agents. As a simple example, owning a mobile phone enables to contact someone instantaneously if the person you want to reach is endowed with the same technology. As a result, the effectiveness of ICT capital is intrinsically related to its own diffusion. We rationalize this line of thought augmenting the production function of each sector with a spillover effect driven by the diffusion of ICT capital. Consistently with this literature and our empirical results, the accumulation of ICT capital is a slow process and the benefits of an ICT technology shock show up in the production functions of ICT-users with lags.\footnote{Notice that differently to \cite{basu2003case} and \cite{basu2007information}, we interpret the general-purpose nature of ICT in the spirit of an endogenous growth model.}

As a last step, we use both our empirical and theoretical results to estimate the key parameters of the model via impulse-response function matching to an ICT technology shock. The key parameters within this set are (i) the elasticity of productivity to ICT capital diffusion, namely the parameter which governs the spillover effect, and (ii) the standard deviation and (iii) persistence of ICT technology shocks. Results consistently point out a positive spillover effect of ICT capital deepening on TFP suggesting that the assumptions made in the theoretical model are supported by our empirical results. We confirm that ICT is a general-purpose technology which enhance productivity of ICT capital users through a spillover effect related to its own diffusion.

\begin{comment}
Our paper is mainly related to three strands of literature. First of all, we link to the literature that investigates the recent slowdown in TFP growth. Our project is motivated by papers such as \cite{oliner2007explaining}, \cite{jorgenson2008retrospective}, \cite{byrne2016does}, \cite{cette2016pre}, and \cite{fernald2017disappointing} who document that the slowdown in TFP growth occurred before the recession implying that the financial crisis per se cannot be its cause.\footnote{In particular, in line with our results, also \cite{fernald2017disappointing} conclude that the pause in the information technology revolution is the main candidate explanation.} Secondly, our project is related to the literature that investigate the effects of investment-specific technology shocks in a multi-sector economy. In particular, throughout the paper our main references will be \cite{greenwood1997long}, \cite{oulton2007investment}, and \cite{fisher2006dynamic}. However, both the empirical exercise and the theoretical model are strictly related to \cite{greenwood2000role}, \cite{basu2010sector}, and \cite{justiniano2011investment}. Finally, this paper is also related to papers that argue that information and communication technology is the current general-purpose technology. Main references will be bresnan, and \cite{basu2007information}.
\end{comment}

The paper is structured as follows. We present empirical results and main robustness checks in Section \ref{section:empirics}. We then present and analyze the 2-sector DSGE model in Section \ref{section:theory}. We estimate via impulse-response matching key parameters of the model and run a series of related experiment in Section \ref{section:experiments}. Concluding remarks, caveats and prospective future research are discussed in Section \ref{section:conclusions}.



%%%%%%%%%%%%%%%%%%%%%%            EMPIRICS                     %%%%%%%%%%%%%%%%%%%
\section{Empirics}\label{section:empirics}

In this section we present our main empirical set of results. Our attempt is to properly identify technological shocks which are only specific to the ICT sector in a Structural VAR context and analyze their impact on key macroeconomic variables.

\section{Main Specification}

In this section we present our main specification where we impose minimal discipline on the identification strategy. It turns out that the set of results presented here are consistent with different robustness checks. Our first-step specification is the following 5-variable VAR 
\begin{equation}
\begin{pmatrix}
TFP_t \\ 
ICT_t \\
GDP_t \\
C_t \\
RP_t \\
\end{pmatrix} = B(L) \begin{pmatrix}
TFP_{t-1} \\ 
ICT_{t-1} \\
GDP_{t-1} \\
C_{t-1} \\
RP_{t-1} \\
\end{pmatrix} + i_t
\end{equation}
where $TFP_t$ is the log-level of Fernald total factor productivity at time $t$, $ICT_{t}$ is the log-level of real information and communication technology investment at time $t$,\footnote{Notice that $ICTI_t$ is defined as the total expenditure at time $t$ in equipment and computer software meant to be used in production for more than an year.} $GDP_t$ is the log-level of real gross domestic product at time $t$, $C_t$ is the log-level of real consumption at time $t$, and $RP_t$ is the log-deviation of ratio between prices of ICT goods and services and the consumer price index (CPI). All the variables refer to the U.S. economy. $B(L)$ is a ($5\times 5$) matrix of lag-operator functions of the same order. Following the Bayesian Information Criterion (BIC), the lag operator functions is one which implies that we regress variables at time $t$ with their own lagged values at $t-1$.\footnote{The Hannan-Quinn Criterion (HQ) suggests to use two lags. Results remains consistent following this second criterion.} Finally, $i_t$ is a ($6\times 1$) vector of correlated innovations where $\Omega = i'_t i_t$.

Using data released on April, 2018 by the Bureau of Economic Analysis (BEA) the real value added of the information sector on real GDP is slightly below $5$\% for the current quarter. Thus, since the share of this sector accounts for a negligible part in the whole economy, as an identification strategy, we assume that an ICT-investment technological change (hereafter ICT shock) is fully orthogonal to current TFP. In addition, in line with theoretical results firstly presented by \cite{greenwood1997long}, we expect that an ICT shock should incentivize sector-specific investment since ICT goods are now easier to produce and cheaper to buy. We transpose this theoretical argument in our empirical VAR by assuming that an ICT shock has a maximal impact effect on ICT investment. 



%%%%%%%%%%%%%%%%%%%%               MODEL                 %%%%%%%%%%%%%%%%%% 
\section{Model}\label{section:theory}


%%%%%%%%%%%%%%%%%%%%               MODEL                 %%%%%%%%%%%%%%%%%% 
\section{Model}\label{section:experiments}




%%%%%%%%%%%%%%%%%%%%               CONCLUSION                 %%%%%%%%%%%%%%%%%% 
\section{Conclusion}\label{section:conclusions}
hjklbhjkl

\bibliographystyle{chicago}
\bibliography{literature}



 %%%%%%%%%%%%%%%%%%           BIBLIOGRAPHY            %%%%%%%%%%%%%%%%%% 

 
\end{document}


