\documentclass[14pt]{article}

% Packages
\usepackage[utf8]{inputenc}
\usepackage[english]{babel}
\usepackage[top=3cm,bottom=3cm,left=3cm,right=3cm,bindingoffset=0mm]{geometry}
\usepackage{amssymb}
\usepackage{amsmath}
\usepackage{tikz}
\usepackage{graphicx}
\usepackage{comment}
\usepackage{rotating}
\usepackage{float}
\usepackage{natbib}
\usepackage{amsthm}
\usepackage{bbm}
\usepackage{thmtools,thm-restate}
\usepackage{hyperref}
\usepackage{extsizes}
\usepackage[font=footnotesize,labelfont=bf]{caption}

% New Options
\newtheorem{prop}{Proposition}
\newtheorem{definition}{Definition}[section]
\newtheorem*{remark}{Remark}
\newtheorem{lemma}{Lemma}
\declaretheorem{proposition}
\linespread{1.3}
\raggedbottom
\font\reali=msbm10 at 12pt

% New Commands
\newcommand{\real}{\hbox{\reali R}}
\newcommand{\realp}{\hbox{\reali R}_{\scriptscriptstyle +}}
\newcommand{\realpp}{\hbox{\reali R}_{\scriptscriptstyle ++}}
\newcommand{\R}{\mathbb{R}}
\DeclareMathOperator{\E}{\mathbb{E}}

\title{ICT and Future Productivity:\\Evidence and Theory of a GPT\thanks{Correspondence: Department of Economics, Boston College, 140 Commonwealth Avenue, Chestnut Hill, MA 02467. Email: brianti@bc.edu (Marco Brianti) and gati@bc.edu (Laura Gati).}}
\author{Marco Brianti \\ {\small Boston College} \and Laura Gati \\ {\small Boston College}}
\date{\today}


\begin{document}



\maketitle
%%%%%%%%%%%%%%%%%%%%             ABSTRACT           %%%%%%%%%%%%%%%%%% 
\small{
\abstract{Information and Communication technology (ICT) is able to explain accelerations in productivity in sectors that are ICT users. We employ Structural VARs to investigate the effects of ICT supply shocks on Total Factor Productivity (TFP) and other macroeconomic variables. In response to this sector-specific supply shock relative prices of ICT goods and services immediately fall, ICT investment rises on impact, and TFP displays a significant delayed and persistent increase. In line with theories of ICT as a general-purpose technology, we analyze a two-sector general equilibrium model in order to rationalize previous results and estimate key parameters via impulse-response matching. We conclude that ICT accumulation is able to enhance productivity through a positive spillover effect which takes into account the overall level of diffusion of ICT capital in the economy.}
}
\newpage



\newpage
%%%%%%%%%%%%%%%%%%%%               INTRODUCTION                 %%%%%%%%%%%%%%%%%% 
\section{Introduction}

Although there is large consensus on the importance of productivity as a driver of economic performances, less agreement is on the underlying sources that enhance its growth. For several years, most of the business-cycle literature purposely decided to avoid such a question by proxying movements in productivity as random shocks.\footnote{\cite{kydland1982time} and \cite{long1983real} are among the first papers which consider productivity shocks on a general equilibrium model.}

However, the robust empirical evidence of the slowdown in productivity right before the great recession is summoning the literature to take a step back and devote more attention on the drivers of medium-term productivity growth.\footnote{See \cite{cette2016pre} and \cite{byrne2016does} among others.}

Along \cite{comin2006medium}, some theoretical contributions rationalize endogenous productivity dynamics by adapting features of endogenous growth models into DSGE models. Following \cite{romer1990endogenous}, most of those papers augment final-good production functions with an expanding composite of intermediate goods produced by the R\&D sector in order to allow for an endogenous rate of adoption of new technologies.\footnote{\cite{bianchi2014growth}, \cite{anzoategui2016endogenous}, and \cite{moran2017innovation} use similar techniques to endogenize growth. In particular, \cite{bianchi2014growth} augment a DSGE  model using a quality ladders model in the vein of \cite{grossman1991quality}. Moreover, \cite{anzoategui2016endogenous} and \cite{moran2017innovation}, similarly to \cite{comin2006medium}, use a model of expanding variety in the vein of \cite{romer1990endogenous}.} Consistent with those previous models, some other papers are exerting effort to show that although research effort is rising, the productivity of the research sector is slowing down.\footnote{\cite{jones2009burden} and \cite{bloom2017ideas} are two important contributions that highlight those facts.}

Motivated by this wave of research, in this paper we empirically investigate which has been one of the main driver of total factor productivity (hereafter TFP) in the last 30 years and how our results can be theoretically rationalized.

We employ Structural VARs to investigate the effects of ICT supply shocks on Total Factor Productivity (TFP) and other macroeconomic variables. In response to this sector-specific supply shock relative prices of ICT goods and services immediately fall, ICT investment rises on impact, and TFP displays a delayed significant and persistent increase. In line with theories of ICT as a general-purpose technology, we analyze a two-sector general equilibrium model in order to rigorously rationalize previous results and estimate key parameters via impulse-response matching. We conclude that ICT accumulation is able to enhance productivity through a positive spillover effect which takes into account the overall level of diffusion of ICT capital in the economy.





%%%%%%%%%%%%%%%%%%%%%%            EMPIRICS                     %%%%%%%%%%%%%%%%%%%
\section{Empirics}



%%%%%%%%%%%%%%%%%%%%               MODEL                 %%%%%%%%%%%%%%%%%% 
\section{Model}




%%%%%%%%%%%%%%%%%%%%               CONCLUSION                 %%%%%%%%%%%%%%%%%% 
\section{Conclusion}
hjklbhjkl

\bibliographystyle{chicago}
\bibliography{literature}



 %%%%%%%%%%%%%%%%%%           BIBLIOGRAPHY            %%%%%%%%%%%%%%%%%% 

 
\end{document}


