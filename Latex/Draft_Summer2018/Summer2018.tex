\documentclass[12pt]{article}

% Packages
\usepackage[utf8]{inputenc}
\usepackage[english]{babel}
\usepackage[top=3cm,bottom=3cm,left=3cm,right=3cm,bindingoffset=0mm]{geometry}
\usepackage{amssymb}
\usepackage{amsmath}
\usepackage{tikz}
\usepackage{graphicx}
\usepackage{comment}
\usepackage{rotating}
\usepackage{float}
\usepackage{natbib}
\usepackage{amsthm}
\usepackage{bbm}
\usepackage{thmtools,thm-restate}
\usepackage{hyperref}
%\usepackage{extsizes}
\usepackage[font=footnotesize,labelfont=bf]{caption}

% New Options
\newtheorem{prop}{Proposition}
\newtheorem{definition}{Definition}[section]
\newtheorem*{remark}{Remark}
\newtheorem{lemma}{Lemma}
\declaretheorem{proposition}
\linespread{1.3}
\raggedbottom
\font\reali=msbm10 at 12pt

% New Commands
\newcommand{\real}{\hbox{\reali R}}
\newcommand{\realp}{\hbox{\reali R}_{\scriptscriptstyle +}}
\newcommand{\realpp}{\hbox{\reali R}_{\scriptscriptstyle ++}}
\newcommand{\R}{\mathbb{R}}
\DeclareMathOperator{\E}{\mathbb{E}}

\title{ICT and Future Productivity:\\Evidence and Theory of a GPT\thanks{Correspondence: Department of Economics, Boston College, 140 Commonwealth Avenue, Chestnut Hill, MA 02467. Email: brianti@bc.edu (Marco Brianti) and gati@bc.edu (Laura G\'ati).}}
\author{Marco Brianti \\ {\small Boston College} \and Laura G\'ati \\ {\small Boston College}}
\date{\today}


\begin{document}



\maketitle

\

\

\
%%%%%%%%%%%%%%%%%%%%             ABSTRACT           %%%%%%%%%%%%%%%%%% 
\small{
\abstract{We employ Structural VARs to investigate the effects of ICT supply shocks on Total Factor Productivity (TFP) and other macroeconomic variables. In response to this sector-specific supply shock, relative prices of ICT goods and services immediately fall, ICT investment rises on impact, and TFP displays a significant delayed and persistent increase. Taking up the view of theories of ICT as a general-purpose technology, we analyze a two-sector general equilibrium model in order to rationalize previous results and estimate spillovers from the stock of ICT via impulse-response matching. We conclude that ICT accumulation is able to enhance productivity through a positive spillover effect which takes into account the overall level of diffusion of ICT capital in the economy.}
}
\newpage



\newpage
%%%%%%%%%%%%%%%%%%%%               INTRODUCTION                 %%%%%%%%%%%%%%%%%% 
\section{Introduction}

Although there is large consensus on the importance of productivity as a driver of economic performance, there is less agreement on the underlying sources of productivity growth. For several years most of the business-cycle literature purposely decided to avoid such a question by proxying movements in productivity as exogenous shocks.\footnote{\cite{kydland1982time} and \cite{long1983real} are among the first papers which consider productivity shocks on general equilibrium models.} However, the robust empirical evidence of the slowdown in productivity right before the Great Recession has led recent literature to take a step back and devote more attention to the drivers of medium-term productivity growth.\footnote{See \cite{cette2016pre} and \cite{byrne2016does} among others.}

Along with \cite{comin2006medium}, theoretical contributions rationalize endogenous productivity dynamics by incorporating features of endogenous growth models in standard models of business cycles. Following \cite{romer1990endogenous}, most of these papers augment final-good production functions with an expanding composite of intermediate goods invented by the Research \& Development (R\&D) sector in order to allow for an endogenous rate of adoption of new technologies.\footnote{\cite{bianchi2014growth}, \cite{anzoategui2016endogenous}, and \cite{moran2017innovation} use similar techniques to endogenize growth. In particular, \cite{bianchi2014growth} augment a DSGE  model using a quality ladders model in the vein of \cite{grossman1991quality}. Moreover, \cite{anzoategui2016endogenous} and \cite{moran2017innovation}, similarly to \cite{comin2006medium}, use a model of expanding variety in the vein of \cite{romer1990endogenous}.} Guided by the prediction of such theoretical work that R\&D developments matter for growth, other papers attempt to provide empirical evidence of a slowdown in the productivity of the R\&D sector. Specifically, they show that although research effort keeps rising, the rate of new ideas and discoveries is slowing down.\footnote{\cite{jones2009burden} and \cite{bloom2017ideas} are two important contributions that highlight these facts.}

Motivated by this wave of research, this paper follows a different path and argues that Information and Communication Technology (hereafter ICT) plays an important role in driving medium-term productivity in sectors that are ICT users. Our contribution is twofold. First, we provide robust empirical evidence to show that current rises in ICT investment explain significant and persistent increases in future Total Factor Productivity (hereafter TFP). Second, we analyze a standard theoretical framework in order to both rationalize and draw conclusions from our empirical results. 

In the empirical section, we identify technological shocks specific to the ICT sector in a Structural VAR context.\footnote{An interesting paper related to our empirical work is \cite{jafari2012impact}. The authors identify ICT shocks as a potential driver of the Iranian business cycle using a completely different identification strategy and obtaining qualitatively different results.} Our multivariate system includes three key variables: TFP, ICT investment (hereafter ICTI), and relative prices (hereafter RP). ICTI is defined as the total expenditure in equipment and computer software meant to be used in production for more than an year. Thus, an increase in ICTI is ICT capital deepening. RP is the ratio between the price of the ICT good and the price level in the overall economy. 

We use two identifying restrictions in order to back out an ICT technology shock. First, we expect it to be orthogonal to the current productivity of all the other sectors. Since the share of the ICT sector accounts for a negligible part in the whole economy, ICT shocks should have an approximately zero effect on TFP on impact. Thus, our first restriction will be a zero-impact restriction on TFP after an ICT shock. Moreover, following \cite{greenwood1997long} and \cite{fisher2006dynamic}, we rely on RP and ICTI because we expect that a sectoral technology shock should decrease its relative prices and enhance expenditure in the underlying sector. For theoretical reasons discussed in more detail below, we do not impose any restriction on RP; instead, we let the ICT shock maximize the impact response of ICTI.\footnote{However, as suggested by both \cite{greenwood2000role} and \cite{basu2010sector}, we are aware that conditioning our identifying restrictions only on the direction of RP does not properly measure for technological changes between sectors. This is the main reason why we never impose the direction of RP as an direct identifying condition.} In response to this shock, ICTI rises on impact and remains significant for several quarters. RP persistently and significantly declines for more than two years, suggesting that we are indeed identifying the correct sectoral shock. Our main result is that TFP, restricted not to respond on impact, rises after few quarters and remains significant and stable for at least 5 years, despite the tiny size of the ICT sector relative to the economy. 

Although our results are robust over different specifications, an important critique to our empirical strategy is reverse causality coming from news on future TFP. As suggested by the news-shock literature, the positive reaction of ICTI on impact may be triggered by signals related future increases in TFP and not to contemporaneous ICT technological improvements. In other words, our identification strategy may confound our shock of interest with a news shock which contemporaneously enhances investment in ICT capital goods. We address this issue by providing a series of alternative identification strategies which we show deliver the same time series of ICT innovations as our initial specification. The heart of these robustness checks is sequential identification of news and ICT shocks: we firstly identify a news shock in the spirit of \cite{barsky2011news}, and subsequently we identify our sectoral ICT shock using the previous identification strategy. Encouragingly, controlling for signals regarding future movements in TFP does not affect our results. We view this as strengthening our statement relating movements in current ICT technology to future TFP. 


In order to understand the economics behind our empirical results, we then analyze a two-sector DSGE model which allows ICT to be the general purpose technology (hereafter GPT) of the whole economy. There are two main justifications for interpreting ICT as a GPT. On the one hand, there is a vast literature that makes a case for the general-purpose nature of ICT capital.\footnote{See for example \cite{oliner2000resurgence} and \cite{stiroh2002information} amongst others.} On the other hand, the small size of the share of the ICT sector both in overall investment and overall output makes our results of a strong and persistent TFP increase after an ICT shock hard to interpret in absence of an additional force such as an externality coming from the general-purpose property of ICT capital.

In our model, both sectoral production functions are fed with three inputs: (i) labor, supplied by households, (ii) hard capital, produced by the final sector, and (iii) ICT capital, produced by the ICT sector. As explained by both \cite{basu2003case} and \cite{basu2007information}, a GPT should be able to enhance accelerations in productivity in sectors that are users of the underlying technology. We then interpret ICT as the GPT of the last 30 years of the U.S. economy assuming that exogenous technological changes in the ICT sector are able to affect economy-wide productivity above the direct effect coming from the technology increase itself.\footnote{A clarification is in place here. In a two-sector model, the overall residual productivity is a convolution of the two exogenous productivities. Thus sectoral productivity changes trivially show up in overall productivity. Our assumption of ICT capital being a GPT means that overall productivity responds more to an ICT-sector-specific technology shock than warranted by this shock directly. We address this question in detail in the main text in Section \ref{section:empirics}.} In particular, when an ICT technology shock arrives, both sectors accumulate ICT capital since it is easier to produce and cheaper to buy. This ICT capital deepening consequently enhances the productivity of both sectors by means of a spillover coming from the accumulated ICT capital stock. 

Since the purpose of ICT capital is to improve information sharing, the quality and speed of communication mainly depends on the diffusion of these technologies among agents. As a simple example, owning a mobile phone enables one to contact another person instantaneously only if the other person is also endowed with the same technology. As a result, the effectiveness of ICT capital is intrinsically related to its own diffusion. This line of thought is what leads us to augment the production function of each sector with a spillover effect capturing the diffusion of ICT capital.  Having a spillover arise from a state variable is also consistent with both the GPT literature above and with our empirical results, since it also leads to model dynamics in which the accumulation of ICT capital is a slow process and the benefits of an ICT technology shock show up in the production functions of ICT-users with lags.\footnote{Notice that differently to \cite{basu2003case} and \cite{basu2007information}, we interpret the general-purpose nature of ICT in the spirit of an endogenous growth model.}

As a last step, we use both our empirical and theoretical results to estimate the key parameters of the model via impulse-response function matching to an ICT technology shock. The key parameters within this set are (i) the elasticity of productivity to ICT capital diffusion, namely the parameter which governs the spillover effect, and (ii) the standard deviation and (iii) persistence of ICT technology shocks. Results consistently point out a positive spillover effect of ICT capital deepening on TFP, confirming that within this class of theoretical models, data supports the existence of spillovers from ICT capital. Thus, our theoretical model suggests to interpret the responses obtained in the empirical section in light of ICT as a general-purpose technology which enhances productivity of ICT capital users through a spillover effect related to its own diffusion.

\begin{comment}
Our paper is mainly related to three strands of literature. First of all, we link to the literature that investigates the recent slowdown in TFP growth. Our project is motivated by papers such as \cite{oliner2007explaining}, \cite{jorgenson2008retrospective}, \cite{byrne2016does}, \cite{cette2016pre}, and \cite{fernald2017disappointing} who document that the slowdown in TFP growth occurred before the recession implying that the financial crisis per se cannot be its cause.\footnote{In particular, in line with our results, also \cite{fernald2017disappointing} conclude that the pause in the information technology revolution is the main candidate explanation.} Secondly, our project is related to the literature that investigate the effects of investment-specific technology shocks in a multi-sector economy. In particular, throughout the paper our main references will be \cite{greenwood1997long}, \cite{oulton2007investment}, and \cite{fisher2006dynamic}. However, both the empirical exercise and the theoretical model are strictly related to \cite{greenwood2000role}, \cite{basu2010sector}, and \cite{justiniano2011investment}. Finally, this paper is also related to papers that argue that information and communication technology is the current general-purpose technology. Main references will be bresnan, and \cite{basu2007information}.
\end{comment}

The paper is structured as follows. We present empirical results and main robustness checks in Section \ref{section:empirics}. We then present and analyze the two-sector DSGE model in Section \ref{section:theory}. We estimate via impulse-response matching key parameters of the model and run a series of related experiments in Section \ref{section:experiments}. Concluding remarks, caveats and prospective future research are discussed in Section \ref{section:conclusions}.



%%%%%%%%%%%%%%%%%%%%%%            EMPIRICS                     %%%%%%%%%%%%%%%%%%%
\section{Empirics}\label{section:empirics}

In this section we present our main empirical results. Our attempt is to properly identify technological shocks which are specific to the ICT sector in a Structural VAR context and analyze their impact on key macroeconomic variables.

\subsection{Main Specification}\label{section:mainSpec}

We start by illustrating our main specification where we impose minimal discipline on the identification strategy. In subsequent sections, we try vastly different alternative empirical strategies. It turns out that the set of results presented here are consistent with the different robustness checks.

\subsubsection{Data}

Our first-step specification is the following 5-variable VAR 
\begin{equation}\label{eq:mainSpecification}
\begin{pmatrix}
TFP_t \\ 
ICTI_t \\
GDP_t \\
C_t \\
RP_t \\
\end{pmatrix} = B(L) \begin{pmatrix}
TFP_{t-1} \\ 
ICTI_{t-1} \\
GDP_{t-1} \\
C_{t-1} \\
RP_{t-1} \\
\end{pmatrix} + i_t
\end{equation}
where $TFP_t$ is the log-level of Fernald total factor productivity at time $t$, $ICTI_{t}$ is the log-level of real information and communication technology investment at time $t$,\footnote{$ICTI_t$ is defined as the total expenditure at time $t$ in equipment and computer software meant to be used in production for more than an year.} $GDP_t$ is the log-level of real gross domestic product at time $t$, $C_t$ is the log-level of real consumption at time $t$, and $RP_t$ is the log-deviation of the ratio between prices of ICT goods and services and the consumer price index (CPI).\footnote{Except for $RP_t$ that is not cointegrated with the remaining variables, we opt for estimating the VAR in levels since it produces consistent estimates of the impulse responses and is robust to cointegration of unknown forms. In particular, as suggested by \cite{hamilton1994time} when there is uncertainty regarding the nature of common trends, estimating the system in levels is considered the conservative approach. The alternative approach is to estimate the system as a vector error correction model (VECM); our results remain very similar also in this case.} Our dataset is quarterly and covers the U.S. economy from 1989:Q1 to 2017:Q1.\footnote{All data are from the BEA except for TFP, which is the series constructed by John Fernald, available on his website.} $B(L)$ is a ($5\times 5$) matrix of lag-operator functions of the same order. Following the Bayesian Information Criterion (BIC), we choose one lag, which implies that we regress variables at time $t$ with their own lagged values at $t-1$.\footnote{The Hannan-Quinn Criterion (HQ) suggests to use two lags. Results remains consistent following this second criterion.} Finally, $i_t$ is a ($5 \times 1$) vector of correlated innovations where $\Sigma = i'_t i_t$.

\subsubsection{Empirical Strategy}\label{section:empiricalstrategy_simple}


Our simplest identification strategy assumes that an ICT-investment technological shock (hereafter ICT shock) has no impact effect on TFP and maximal impact effect on ICTI. We justify these assumptions with both empirical and theoretical arguments. First of all, using data released in April, 2018 by the Bureau of Economic Analysis (BEA) the real value added of the information sector on real GDP is slightly below $5$\% for the underlying quarter.\footnote{There are many ways to quantify the size of the ICT sector, both in terms of what share one considers and how one defines the ICT sector. The number presented in the text refers to the value added share of the ICT sector, defined as in Section \ref{eq:mainSpecification}. The statement that the ICT sector is tiny is however not sensitive to the definition of the sector or to the choice of share. Even with broader definitions, one obtains numbers in the range of 5\%.} Thus, since the share of this sector accounts for a negligible part in the whole economy, we think it safe to assume that an ICT shock is orthogonal to current TFP. In addition, the theory of multi-sector models, firstly presented by \cite{greenwood1997long}, predicts that a productivity increase in one sector should lead to sectoral output becoming cheaper. Thus, we expect that an ICT shock should enhance sector-specific investment since ICT goods are now easier to produce and cheaper to buy. As a result, we expect an ICT shock to have a maximal impact effect on ICTI. 

Using a notation similar to \cite{barsky2011news}, we implement our identification strategy as follows. Let $y_t$ be the $(5 \times 1)$ vector of observables of length $T$ presented above. The reduced-form VAR takes the following form,
$$
y_t = B(L)y_{t-1} + i_t 
$$
Assume now that there exists a linear combination that maps innovations $i_t$ to structural shocks $s_t$,
$$
A_0 s_t = i_t
$$
which implies the structural-form VAR
$$
A_0^{-1}y_t = C(L) y_{t-1} + s_t
$$
where $C(L) = A_0^{-1} B(L)$ and $s_t = A_0^{-1} i_t$. The impact matrix $A_0$ must be such that $\Sigma = A_0 A_0'$. Notice that $A_0$ is not unique since for any $D$ such that $DD' = I$, $\tilde{A}_0 = A_0 D$ satisfies $\Sigma = \tilde{A}_0 \tilde{A}_0'$. 

The matrix of impact responses to all shocks is: % TYPO HERE? either Atilde or A0 D. Checked Barsky, this seems correct but I'm confused. 
$$
\Pi(0) = \tilde{A}_0 D
$$
and is specifically formed by the following elements, denoting the responses of variable $i$ to shock $j$, % same here
$$
\Pi_{i,j}(0) = e_i' \tilde{A}_0 D e_j
$$
where $e_k$ is a selector column vector of the same dimension as $\tilde{A}_0$ with $1$ in the $k$th element and zero elsewhere. In particular, notice that $e_j$ is selecting a specific column of $D$. Let this column be denoted by $\gamma_j$. Using this notation, $\tilde{A}_0\gamma_j$ is the vector of impact responses of all the variable to shock $j$.

Observe from System \ref{eq:mainSpecification} that $TFP_t$ is ordered first and $ICTI_t$ second. Our identification strategy is then the solution to the following problem:
\begin{equation}\label{eq:mainObjective}
\max_{\gamma_j} \Pi_{2,j}(0) = e_2' \tilde{A}_0 \gamma_j
\end{equation}
subject to
\begin{equation}\label{eq:mainZeroTFP}
\Pi_{1,j}(0) = e_1' \tilde{A}_0 \gamma_j = 0, \ \ \text{and}
\end{equation}
\begin{equation}\label{eq:mainOrtho}
\gamma_j' \gamma_j = 1
\end{equation}
where $j$ represents the arbitrary position of the ICT shock. Then, in order to ensure that this identification belongs to the space of possible orthogonalizations of $\Sigma$, the problem is formulated in terms of choosing $\gamma_j$ conditional on any orthogonalization, $\tilde{A}_0$. Objective function \ref{eq:mainObjective} imposes that an ICT shock has a maximal impact effect on ICT investment. Constraint \ref{eq:mainZeroTFP} orthogonalizes current TFP to ICT shocks and Constraint \ref{eq:mainOrtho} satisfies the condition that $\gamma_j$ is derived from an orthogonal matrix $D$.  

\subsubsection{Main Set of Results}

Appendix \ref{section:mainSetResults} shows the estimated impulse responses of System \ref{eq:mainSpecification} to the identified ICT shock. The shaded gray areas are the $90$\% and $95$\% confidence bands from a bias-corrected bootstrap procedure of \cite{kilian1998small} using 2000 simulations. Our main interest, Figure \ref{fig:TFP_main}, shows the impulse response of TFP to an ICT shock. TFP takes around $4$ quarters before displaying a positive and significant effect and reaches its peak of $1.2$\% after $24$ quarters. In Figure \ref{fig:ICT_main}, real ICT investment has a large and positive impact response of almost $2$\% that gets even larger after a quarter. Then, it slowly starts to decay remaining significant for more than $40$ quarters. In Figures \ref{fig:GDP_main} and \ref{fig:C_main}, we present responses of real GDP and real consumption, respectively. Real GDP has a significant impact response of $0.3$\% and reaches its peak of almost $0.5$\% approximately at the same time as TFP. Similarly, real consumption has an impact effect of $0.2$\% with a delayed peak of $0.5$\%. Finally,  Figure \ref{fig:RP_main} depicts that response of relative prices. Despite imposing no restriction here, the response of RP is broadly in line with multi-sector theory. Relative prices drop significantly on impact by $0.4$\% and remain persistently below their own steady state value for almost $9$ years.

In addition, Table \ref{table:vardec} in Appendix \ref{section:vardec} presents forecast error variance decompositions for the ICT shock on each variable. The table shows decompositions on impact, and at a one-, two-, four-, six-, and ten-year horizon. These results are very interesting in their own right, as they hint that the ICT sector may have a particular role for the overall productivity of the economy.

In line with the impulse responses, ICT shocks, which are orthogonal to current productivity, explain nothing of current TFP fluctuations. At a ten-year horizon, however, this fraction is over a third! This crucial result is a strong sign that the ICT sector matters for overall productivity over and beyond the direct contribution of its sectoral productivity (which, as we have seen, is marginal). At the same time, ICT shocks drive almost the whole variation in ICT investment on impact, with a slow decay over time to below $50$\% after 10 years. Interestingly, both output and consumption have a remarkable reaction on impact: $26$\% and $19$\%, respectively. Moreover, this effect tends to increase reaching $40$\% in both cases at the maximal horizon. Finally, ICT shocks only account for a small fraction of movements in relative prices. The fraction of fluctuations explained on impact is only $6$\% with a peak of $14$\% between four and six years. Overall, these variance decompositions all seem to suggest that the ICT shock leads to complex forms of reallocation and accumulation that take place over the medium run in the economy. Our structural model of Section \ref{section:theory} provides a theoretical framework to think about these dynamics and to give them an economic interpretation.

\subsection{Controlling for News Shocks}

In this section, we present a set of robustness checks aimed to show that our previous results are not driven by future signals regarding exogenous productivity. The main concern against the specification of \ref{section:mainSpec} is reverse causality coming from the presence of news about future TFP that is not accounted for. Indeed, the news-shock literature warns that the positive reaction of ICTI on impact may be triggered by signals related to future increases in productivity and not to a contemporaneous ICT shock. In other words, our identification strategy may confound our shock of interest with a news shock which contemporaneously enhances investment in ICT capital goods. We address this issue by providing two main alternative identification strategies which we show deliver the same time series of ICT shocks as our initial specification. 

As a first-pass test, % cite Barsky Basu Lee and Forni or whatevs
we check if our results are robust once we remove all the forward looking variables whose fluctuations may be unrelated to sector-specific technological changes: consumption and output. Technically speaking, consumption and output may Granger-cause future TFP for reasons which are orthogonal to ICT shocks. For example, the forward-looking nature of consumption may provide the VAR with information regarding future changes in TFP not related to an ICT shock, which our identification strategy is not able to filter out. Our second test is running a VAR in which we identify both ICT shocks and news shocks. In particular, we apply a sequential identification where we first identify a news shock in the spirit of \cite{barsky2011news}, and subsequently we identify our sectoral ICT shock using our original identification strategy. This second strategy has the specific purpose of filtering out all the current movements in forward-looking variables related to future fluctuations of TFP which are not related to current ICT shocks, as such fluctuations are all captured by the identified news shock.

As is illustrated in the rest of this section, both our robustness checks recover the same series for structural ICT shocks as our initial specification, and thus leave our results unaffected. We therefore conclude that controlling for signals regarding future movements in productivity does not affect our identified ICT shocks, and thus confirm the causal relation between current ICT investment and future TFP.

\subsubsection{Removing Forward-Looking Variables}

In this part of the paper, we employ a first-pass test where we attempt to show that our ICT shock is not confounded with future signals of TFP unrelated to the ICT sector. 

As discussed by \cite{sims2012news}, \cite{forni2014sufficient}, and \cite{barsky2015whither} the presence of forward-looking variables in the VAR is a necessary condition to correctly identify a news shock. In particular, since the univariate TFP process cannot predict future TFP,\footnote{It is well-accepted that TFP growth can be approximated as a white noise.} a news shock can be identified if there are variables that granger-cause TFP inside the VAR. For example, \cite{beaudry2006stock} use current movements in stock prices to predict future productivity, but also contemporary fluctuations in consumption, hours, investment, may implicitly reflect information regarding future technical change.

Thus, as a simple test we propose to redo the exercise described in Section \ref{section:mainSpec} without two forward-looking macroeconomic variables not directly related with the ICT sector: consumption and output.\footnote{Notice that output implicitly reflect the behavior of hours and investment once we control for consumption.} If we are actually confounding news shocks with our ICT shock series then we would expect to see some differences in this second estimate. In other words, once our VAR is deprived by a source of information useful to recover a news shock, then the confounded part should either decreases or cancels out. 

As hinted above, in this section we estimate the following 3-dimensional system
\begin{equation}\label{eq:noForVariables}
\begin{pmatrix}
TFP_t \\ 
ICTI_t \\
RP_t \\
\end{pmatrix} = B(L) \begin{pmatrix}
TFP_{t-1} \\ 
ICTI_{t-1} \\
RP_{t-1} \\
\end{pmatrix} + i_t
\end{equation}
where  $TFP_t$, $ICTI_{t}$, and $RP_t$ have the same meaning and transformation as in System \ref{eq:mainSpecification}. As before, dataset is quarterly and covers the U.S. economy from 1989:Q1 to 2017:Q1. Finally, according to the Bayesian Information Criterion (BIC), we choose one lag.\footnote{The Hannan-Quinn Criterion (HQ) suggests to use two lags. Results remains consistent following this second criterion.} Using the same identification strategy presented in \ref{section:empiricalstrategy_simple}, we recover a series of ICT shocks which is correlated at $95$\% with the one estimated in 5-dimensional system. In particular, in Appendix \ref{section:removing_FL_Var}, Figure \ref{fig:shockSeries_removing} shows ICT shock series for both the 3-dimensional system and 5-dimensional one. In line with the correlation coefficient, the two series almost overlap each others suggesting that our initial result is not driven by the forward-looking power of consumption and output on TFP. 

However, although this result is encouraging, this first-pass test has obviously a weak power because the jump of ICTI on impact may be triggered by future exogenous signal of TFP. In this case, ICTI would be itself the source to granger-cause TFP and our ICT shock would be upward biased by news. To amend this deeper concern, we then need to employ a more complicated strategy presented in the next section. 


\subsubsection{Sequential Identification Strategy}

In this part, we present an alternative procedure to rule out the presence of news from our empirical result. In this second test we run a VAR where we identify both ICT shocks and news shocks. We apply a sequential identification where we first identify a news shocks in the spirit of \cite{barsky2011news}, and subsequently we identify ICT shocks using our original identification strategy. Here we are seriously attempting to filter out future signals unrelated to current ICT shocks. Specifically, in the first step the news shock is identified as the shock orthogonal to current TFP that best explains its future movements. Then, as a second step, the ICT shock is identified as the shock orthogonal to current TFP that maximizes the impact effect on ICT investment. 

However, we need to clarify that in order to correctly employ this procedure, we need to take care of a second issue. As correctly mentioned by \cite{barsky2011news}, \textit{``A general objection to our emprical approach would be that a number of structural shocks, which are not really news in the sense defined in the literature, might affect a measure of TFP in the future without impacting it immediately. Among these shocks might be research and development shocks, investment specific shocks, and reallocative shocks. Our identification (and any other existing VAR identifications) would obviously confound any true news shock with these shocks.''} In other words, by naively employing the procedure presented above, any ICT shocks would be fully captured as a news shock since it is orthogonal to current TFP and explains its future movements. 

Thus, in order to avoid the opposite mistake, namely confounding ICT shocks into news shocks, we need to augment the first step with a sort of control ad hoc for our case. To formally discipline this additional identification assumption we need to design a two-sector model in the spirit of \cite{greenwood1997long}.\footnote{In particular, the model presented in Section \ref{section:theory} is closely related to \cite{oulton2007investment}.} In particular, we will mostly rely on the effect of an sector-neutral technology shock on relative prices to constraint the first step in order to avoid to capture an ICT shock into a news shock.

\begin{prop}\label{prop:priceRestriction}
In a two-sector economy, if both production functions are identical then any economy-wide productivity shocks do not affect relative prices between the two sectors.
\end{prop}
\begin{proof}
	See Appendix \ref{section:proofpriceRestriction}.
\end{proof}

Proposition \ref{prop:priceRestriction} implies that a future sector-neutral technological change (the so-called news shock) never affects relative prices between sectors. Obviously, this statement is correct under the extreme assumptions that (i) the shock is expected to have the same impact on the two sectors and (ii) production functions are identical to each others. If those assumptions hold in the data, then the correct additional restriction would be to shut down to zero the impulse response of relative prices to a news shock. However, this restriction would obviously penalize the identification of news shocks since it is unlikely that assumptions (i) and (ii) perfectly hold in the data. We then opt for a more moderate strategy where we basically impose that the response of relative prices to a news shock should be relatively much smaller than the response observed by an ICT shock. Practically speaking, we arbitrarily set the impulse response of relative prices to be zero on some specific horizon rather than always. In particular, although results are robust to different horizon restrictions, our favorite specification is the one which set relative prices to have zero effect in the first two periods. 

In this case, our specification is the following 6-variable VAR,
\begin{equation}\label{eq:newsSpecification}
\begin{pmatrix}
TFP_t \\ 
SP_t \\
ICTI_t \\
GDP_t \\
C_t \\
RP_t \\
\end{pmatrix} = B(L) \begin{pmatrix}
TFP_{t-1} \\ 
SP_{t-1} \\
ICTI_{t-1} \\
GDP_{t-1} \\
C_{t-1} \\
RP_{t-1} \\
\end{pmatrix} + i_t
\end{equation}
where $TFP_t$, $ICTI_t$, $GDP_t$, $C_t$, and $RP_t$ have the same meaning and transformation as in System \ref{eq:mainSpecification}. Moreover, $SP_t$ represents the log-transformation of Standard \& Poor's 500 stock prices added to provide an additional forward-looking variable to the system.\footnote{As mentioned before, following \cite{sims2012news}, \cite{forni2014sufficient}, and \cite{barsky2015whither} the presence of forward-looking variables in the VAR is a necessary condition to correctly identify a news shock.} As before, dataset is quarterly and covers the U.S. economy from 1989:Q1 to 2017:Q1 and following the BIC and HQ criteria we opt for one lag.

Using a similar notation in Section \ref{section:empiricalstrategy_simple}, assume for notation simplicity that $B(L) = B$ allows only of one lag. Define the impulse response of variable $i$ to the identified shock $j$ at time $t$ as,\footnote{We also suppress the constant to simplify the notation. Identification strategy can obviously be applied for $B(L)$ allowing for more than one lag and with the constant.}
$$
\Pi_{i,j}(t) = e_i' B^t \tilde{A}_0 D e_j = e_i' B^t \tilde{A}_0 \gamma_j
$$
and its variance decomposition up to time $h$
$$
\Omega_{i,j}(h) = \frac{ \sum_{t=0}^h e_i' B^t \tilde{A}_0 D e_j e_j' D' \tilde{A}_0' B'^t e_i } {e_i'( \sum_{\tau = 0}^H B^t \Sigma B'^t )e_i} = \frac{ \sum_{t=0}^h e_i' B^t \tilde{A}_0 \gamma_j \gamma_j' \tilde{A}_0' B'^t e_i } {e_i'( \sum_{\tau = 0}^H B^t \Sigma B'^t )e_i}
$$
Given the following notation and let notice that TFP is ordered first and ICTI second, the identification strategy can be summarized as follows,

\subsubsection*{Step 1 - Identification of $\gamma_{news}$}
$$
\max_{\gamma_{news}} \Omega_{1,news}(h) = \frac{ \sum_{t=0}^h e_1' B^t \tilde{A}_0 \gamma_{news} \gamma_{news}' \tilde{A}_0' B'^t e_1 } {e_1' ( \sum_{\tau = 0}^H B^t \Sigma B'^t )e_1}
$$
subject to
$$
\Pi_{1,news}(0) = 0,
$$
$$
\Pi_{6,news}(0) = \Pi_{6,news}(1) = 0, \ \ \text{and}
$$
$$
\gamma_{news} \gamma_{news}' = 1.
$$
where the first constraint represents the zero-impact restriction of news on TFP. Moreover, the second constraint represents the arbitrary restriction to have a response of relative prices to news relatively small.\footnote{Our identification strategy is robust over different restrictions to relative prices. Results hold even if we set the zero restriction on other horizons.} Finally, the last constraint imposes that $\gamma_{news}$ should be a column derived from the orthogonal matrix $D$. 


\subsubsection*{Step 2 - Identification of $\gamma_{ICT}$}
$$
\max_{\gamma_{ICT}} \Pi_{2,ICT}(0) =  e_2' \tilde{A}_0 \gamma_{ICT} 
$$
subject to
$$
\Pi_{1,ICT}(0) = 0,
$$
$$
\gamma_{news} \gamma_{ICT}' = 0, \ \ \text{and} \ \ \gamma_{ICT} \gamma_{ICT}' = 1.
$$
where the first constraint represents the zero-impact restriction of news on TFP and the last constraint imposes that both $\gamma_{news}$ and $\gamma_{ICT}$ should be two different columns derived from the orthogonal matrix $D$.







%%%%%%%%%%%%%%%%%%%%               MODEL                 %%%%%%%%%%%%%%%%%% 
\section{Model}\label{section:theory}


%%%%%%%%%%%%%%%%%%%%               MODEL                 %%%%%%%%%%%%%%%%%% 
\section{Experiments}\label{section:experiments}




%%%%%%%%%%%%%%%%%%%%               CONCLUSION                 %%%%%%%%%%%%%%%%%% 
\section{Conclusion}\label{section:conclusions}
hjklbhjkl

\bibliographystyle{chicago}
\bibliography{literature}

 %%%%%%%%%%%%%%%%%%           BIBLIOGRAPHY            %%%%%%%%%%%%%%%%%% 

\newpage

 %%%%%%%%%%%%%%%%%%           Appendices            %%%%%%%%%%%%%%%%%% 

\appendix 

\section{Main Set of Results}\label{section:mainSetResults}

	\begin{figure}[h!]
		\begin{center}
\includegraphics[scale=0.35]{MainFigures/fig_ICT_Shock_on_TFP_empirical_noH}
		\caption{Empirical impulse response of TFP to an ICT shock. The red solid lines are the estimated impulse responses to our ICT shock. The shaded dark gray area and the shaded light gray area are the $90$\% and $95$\% confidence intervals, respectively, from 2000 bias-corrected bootstrap replications of the reduced-form VAR. The horizontal axes refer to forecast horizon and the units of the vertical axes are percentage deviations.}
		\label{fig:TFP_main}
	\end{center}
\end{figure}

\newpage




	\begin{figure}[h!]
		\begin{center}
		\includegraphics[scale=0.35]{MainFigures/fig_ICT_Shock_on_Real_ICT_Investment_empirical_noH}
		\caption{Empirical impulse response of real ICT investment to an ICT shock. The red solid lines are the estimated impulse responses to our ICT shock. The shaded dark gray area and the shaded light gray area are the $90$\% and $95$\% confidence intervals, respectively, from 2000 bias-corrected bootstrap replications of the reduced-form VAR. The horizontal axes refer to forecast horizon and the units of the vertical axes are percentage deviations.}
		\label{fig:ICT_main}
	\end{center} 
	\end{figure}

\newpage


	\begin{figure}[h!]
	\begin{center}
		\includegraphics[scale=0.35]{MainFigures/fig_ICT_Shock_on_Real_GDP_empirical_noH}
		\caption{Empirical impulse response of real Gross Domestic Product to an ICT shock. The red solid lines are the estimated impulse responses to our ICT shock. The shaded dark gray area and the shaded light gray area are the $90$\% and $95$\% confidence intervals, respectively, from 2000 bias-corrected bootstrap replications of the reduced-form VAR. The horizontal axes refer to forecast horizon and the units of the vertical axes are percentage deviations.}
		\label{fig:GDP_main}
	\end{center} 
\end{figure}

\newpage


	\begin{figure}[h!]
	\begin{center}
		\includegraphics[scale=0.35]{MainFigures/fig_ICT_Shock_on_Real_Consumption_empirical_noH}
		\caption{Empirical impulse response of real consumption to an ICT shock. The red solid lines are the estimated impulse responses to our ICT shock. The shaded dark gray area and the shaded light gray area are the $90$\% and $95$\% confidence intervals, respectively, from 2000 bias-corrected bootstrap replications of the reduced-form VAR. The horizontal axes refer to forecast horizon and the units of the vertical axes are percentage deviations.}
		\label{fig:C_main}
	\end{center} 
\end{figure}

\newpage


	\begin{figure}[h!]
	\begin{center}
		\includegraphics[scale=0.35]{MainFigures/fig_ICT_Shock_on_Relative_Price_empirical_noH}
		\caption{Empirical impulse response of relative price to an ICT shock. The red solid lines are the estimated impulse responses to our ICT shock. The shaded dark gray area and the shaded light gray area are the $90$\% and $95$\% confidence intervals, respectively, from 2000 bias-corrected bootstrap replications of the reduced-form VAR. The horizontal axes refer to forecast horizon and the units of the vertical axes are percentage deviations.}
		\label{fig:RP_main}
	\end{center} 
\end{figure}




\newpage

 
 
 \section{Variance Decomposition}\label{section:vardec}

 
 
 	\begin{table}[h!]
 		\begin{center}
 \begin{tabular}{lcccccccccc}
\hline
 	& $h = 1$ & $h = 4$ & $h = 8$ & $h = 16$ & $h = 24$ & $h = 40$ \\
 	\hline
TFP &  0       &  0.0023  &  0.0194 &   0.1088 &   0.2273  &  0.3382 \\
ICT Investment &  0.9997  &  0.9038  &  0.7964 &   0.6320 &   0.5310  &  0.4371 \\
Real GDP &  0.2620  &  0.3061  &  0.3486 &   0.3936 &   0.4046  &  0.3881 \\
Real Consumption &  0.1952  &  0.2638  &  0.3219 &   0.3931 &   0.4188  &  0.4064 \\
Relative Prices &  0.0618  &  0.0967  &  0.1276 &   0.1511 &   0.1516  &  0.1467 \\	
\hline
 	\end{tabular}
  		\caption{The letter $h$ denotes the forecast horizon. The numbers refer to the fraction of the forecast error variance of each variable at various forecast horizons to the identified ICT shock}
  \label{table:vardec}
  \end{center}
 \end{table}

\newpage

 \section{Removing Forward-Looking Variables}\label{section:removing_FL_Var}
 
 
 	\begin{figure}[h!]
 	\begin{center}
 		\includegraphics[scale=0.35]{MainFigures/Removing_FLVariables}
 		\caption{ICT shock series using the empirical strategy presented in \ref{section:empiricalstrategy_simple}. Blue solid line represents shock series for the 5-dimension system \ref{eq:mainSpecification} presented in \ref{section:mainSpec}. Red dotted line represents shock series for the 3-dimensional system presented in \ref{section:removing_FL_Var}.}
 		\label{fig:shockSeries_removing}
 	\end{center} 
 \end{figure}

\newpage

\section{Proof of Proposition \ref{prop:priceRestriction}}\label{section:proofpriceRestriction}

Prova


 


 
\end{document}


